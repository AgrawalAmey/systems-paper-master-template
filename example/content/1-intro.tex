\section{Introduction}


% Start with applications and why long context matters
Large language models with million-token context windows are transforming how we interact with information -- enabling large-scale document analysis, multi-hour video understanding \cite{gemini,llama4,qwen2.5}, and autonomous coding agents \cite{anthropic2025claudecode, google2025geminicli}.
However, deploying these models in production poses a significant challenge: real-world workloads combine both long and short requests. A single service must handle everything from 100-token chat messages to 10M-token document processing, often from the same users within the same session.


% \begin{figure}[t!]
%     \centering
%     \includegraphics[width=\linewidth]{figures/banner_exp_ttft_cdf_moneta.pdf}
%     % \vspace{-0.1in}
%     % \caption{
%     %     Mnemosyne based of latency with Llama3-8B on \amey{some people might get offended by the "just" xD }just 128 H100s (16 servers).
%     % }
%     \caption{
%         Impact of long-context requests on TTFT latency for \llamaS{} inference using 16 A100 GPUs.
%         \ls{}~\cite{2024loongserve} shows 30-174\myx{} higher latency due to its coarse-grained space-sharing, underscoring the lack of fine-grained time-sharing and the resulting head-of-line blocking between long- and short-context requests.
%         \inigo{Show the TTFT over time for \ls{} and Moneta.
%         It should include the load (prefill/decode tokens per seconds?).
%         Add a line for the P90 or the full CDF.}
%     }
%     \label{fig:banner}    
%     % \vspace{-0.1in}
% \end{figure}

%     \includegraphics[
%     width=\linewidth, % This will scale the *remaining* right half to the full line width
%     trim={220bp 0 0 0}, % Trim 250bp from left, 0 from others
%     clip % Apply the trim
% ]{figures/banner_exp_ttft_cdf_moneta.pdf}

\begin{figure}[t!]
    % \centering
    \begin{subfigure}{0.48\linewidth}    \includegraphics[width=\linewidth]{figures/banner_figure_bar.pdf}
        \vspace{0em}
        \caption{
            TTFT distribution where state-of-the-art baseline shows 30-174\myx{} higher latency due to lack of preemptive scheduling.
        }
        \label{fig:banner:bars}
    \end{subfigure}
    %
    \begin{subfigure}{0.48\linewidth}          
    \includegraphics[width=0.88\linewidth]{figures/ttft_comparison_scatter_medha.pdf}
        \caption{
            TTFT over time showing severe convoy effect for \ls{} where short requests get stuck behind long requests.
        }
        \label{fig:banner:hol}
    \end{subfigure}
    \caption{
        Impact of long-context requests on TTFT for \llamaS{} inference using 16 A100 GPUs with \ls{}~\cite{2024loongserve} and \sysname{} at 0.75 QPS.
        %underscoring the lack of fine-grained time-sharing and the resulting head-of-line blocking between long- and short-context requests.
    }
    \label{fig:banner}    
    % \vspace{-0.1in}
\end{figure}


% Why this mixture creates problems
\textbf{Motivation.}
This mix of request lengths to the same model instance creates extreme computational heterogeneity due to the quadratic complexity of self-attention~\cite{attentionpaper}.
A 100K-token request is not 100\myx but approximately 10,000\myx more computationally expensive than a 1K-token request.
When these requests share the same serving infrastructure, we run into severe performance degradation due to: the \textit{convoy effect}~\cite{ostep,dino}, where long-running requests block shorter ones, resulting in poor system responsiveness.
As shown in \Cref{fig:banner}, even with just 5\% long requests in the workload, state-of-the-art systems like \ls~\cite{2024loongserve} experience 30\myx median latency increases and 174\myx tail latency degradation for short requests.

% Why current solutions fail
Context Parallelism (CP)~\cite{liu2023ring,brandon2023striped} has made it possible to distribute long-context processing across hundreds of GPUs, enabling effective training for long-context models.
\ls~\cite{2024loongserve} and Yang et al. \cite{yang2023cp}
adapted these techniques for inference by introducing elasticity to context parallelism, dynamically adjusting the degree of parallelism based on request length to improve resource efficiency.
However, these approaches fundamentally lack preemptability --- once a long request begins processing, it cannot be interrupted until completion.
This non-preemptive execution model inevitably results in convoy effects in heterogeneous workloads.
\Cref{fig:banner:hol} demonstrates this problem empirically: \ls exhibits severe latency spikes lasting for 100s of seconds whenever long requests block the system.
These spikes occur because arriving short requests must wait for the entire prefill computation of the long request to complete --- which can take several minutes.

% Core insight: Need preemption
The convoy effect is a well-studied problem in operating systems with a known solution: preemptive scheduling.
Chunked prefills \cite{agrawal2024taming} provide a natural mechanism for preemptable prefill computation.
However, applying preemption to LLM inference faces three barriers that have discouraged its adoption.
First, chunking long prefills are considered inefficient due to\textbf{\textit{ repeated KV-cache reads.}}
Second, batching decodes requests with chunked prefills for long requests results in \textbf{\textit{high decode latency}} that degrades user experience.
Finally, existing context parallelism techniques for long-context inference are \textbf{\textit{fundamentally incompatible with chunked execution}}.

\begin{figure}[t!]
    \centering
    \includegraphics[width=\linewidth]{figures/scheduling/convoy-effect-medha.pdf}
    \caption{Impact of preemption on convoy effect. Non-preemptive scheduling (top) blocks short requests B and C behind long request A, causing deadline violations. Preemptive scheduling (bottom) interleaves execution through chunking, eliminating convoy effect while maintaining throughput.}
    \label{fig:convoy}
\end{figure}

% Our approach: Making preemption practical
\textbf{Our work.}
\sysname makes preemptive long-context inference practical by systematically addressing each barrier.
We demonstrate that KV-cache read amplification is a non-issue for modern architectures -- chunks as small as 40 tokens achieve near-optimal efficiency due to high arithmetic intensity in grouped-query attention.
We introduce \textit{adaptive chunking}, which dynamically adjusts chunk sizes as the computational bottleneck shifts from MLP to attention operations, maintaining both high throughput and predictable latency.
We develop two parallelism strategies compatible with preemption:
\textbf{\textit{Stream Pipeline Parallelism (SPP)}} accelerates prefills by pipelining chunks across stages, while \textbf{\textit{KV-Cache Parallelism (KVP)}} bounds decode latency by distributing attention computation.

% Scheduling contribution
To effectively leverage the preemptable prefills, we introduce Length-Aware Relative Slack (LARS), a deadline-aware  scheduling policy designed to explicitly tackle the heterogeneous nature of long-context inference. Unlike traditional policies that either cause convoy effects (First Come First Serve - FCFS) or starvation (Earliest Deadline First - EDF, Least Remaining Slack - LRS), LARS ensures both short and long requests meet their deadlines by pushing completions toward their SLO boundaries -- maximizing schedule robustness against unpredictable arrivals. Furthermore, we introduce a  dynamic batch packing algorithm that creates batches that maximally utilize GPU compute by co-locating complementary prefill chunks and decode requests while respecting strict time budgets.

\sysname unifies these techniques in a unified serving system that scales to multi-million token requests while maintaining high throughput and low latency. In summary, we make the following contributions in this paper:
\begin{itemize}[leftmargin=*,noitemsep,topsep=1pt]
    \item We identify and quantify the convoy effect in long-context inference arising due to request-length heterogeneity in current non-preemptive systems.

    \item We make chunked prefills viable for preemptive long-context inference by systematically addressing their perceived inefficiencies: introducing adaptive chunking to balance throughput and latency, and developing Stream Pipeline Parallelism (SPP) and KV-Cache Parallelism (KVP) as preemption-compatible parallelism strategies.

    \item We propose Length-Aware Relative Slack (LARS), a scheduling policy that prevents convoy effects and starvation by  accounting for workload heterogeneity and user SLOs.

    \item We implement \sysname with optimized kernels and scheduler design, demonstrating 5.7\myx higher throughput and up to 174\myx lower tail latency than state-of-the-art non-preemptive systems on real long-context workloads.
\end{itemize}
