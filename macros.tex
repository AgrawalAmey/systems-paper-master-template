% ============================================================
% COMMON MACROS
% Project-wide commands, abbreviations, and author comments.
% ============================================================

% --- System Name ---
% Derived from \sysnameplain in config.tex.
\newcommand{\sysname}{\textsc{\sysnameplain}\xspace}

% --- Standard Abbreviations ---
\newcommand{\ie}{\textit{i.e.,}\xspace}
\newcommand{\eg}{\textit{e.g.,}\xspace}
\newcommand{\etal}{\textit{et al.}\xspace}
\newcommand{\etc}{\textit{etc.}\xspace}
\newcommand{\wrt}{w.r.t.\xspace}
\newcommand{\aka}{a.k.a.\xspace}
\newcommand{\cf}{{cf.}\xspace}

% --- Section References ---
\newcommand{\sref}[1]{\S\ref{#1}}
\newcommand{\figref}[1]{Figure~\ref{#1}}
\newcommand{\tabref}[1]{Table~\ref{#1}}
\providecommand{\algref}[1]{Algorithm~\ref{#1}}
\newcommand{\eqnref}[1]{Eq.~\ref{#1}}

% --- Formatting Helpers ---
\newcommand{\vheading}[1]{\vspace{4pt}\noindent\textbf{#1.}\xspace}
\newcommand{\heading}[1]{\noindent\textbf{#1.}\xspace}
\newcommand{\myparagraph}[1]{\vspace{\smallskipamount}\noindent\textbf{#1.\xspace}}
\newcommand{\mycaption}[2]{\caption{\textbf{#1.} {#2}}}
\newcommand{\ra}[1]{\renewcommand{\arraystretch}{#1}}

% --- Math Operators ---
\DeclareMathOperator*{\argmax}{arg\,max}
\DeclareMathOperator*{\argmin}{arg\,min}

% --- Table Symbols ---
\newcommand{\greencheck}{\textcolor{ForestGreen}{\checkmark}\xspace}
\newcommand{\redcross}{\textcolor{red}{$\times$}\xspace}
\newcommand{\myx}{$\times$\xspace}
\newcommand{\greenup}{\textcolor{ForestGreen}{\textbf{$\uparrow$}}\xspace}
\newcommand{\reddown}{\textcolor{red}{\textbf{$\downarrow$}}\xspace}
\newcommand{\rot}[1]{\rotatebox{90}{#1}}

% --- Figure Part Labels ---
\newcommand{\figleft}{{\em (Left)}\xspace}
\newcommand{\figright}{{\em (Right)}\xspace}
\newcommand{\figtop}{{\em (Top)}\xspace}
\newcommand{\figbottom}{{\em (Bottom)}\xspace}

% --- Compact Lists (for tight spaces) ---
\newcommand{\squishlist}{\begin{itemize}[leftmargin=*,noitemsep,topsep=0pt]}
\newcommand{\squishend}{\end{itemize}}

% --- Callout / Insight Boxes ---
% Usage: \begin{insightbox} ... \end{insightbox}
\newenvironment{insightbox}{%
    \def\FrameCommand{\fboxrule=0.8pt \fboxsep=4pt \fcolorbox{black!60}{black!3}}%
    \MakeFramed{\advance\hsize-\width \FrameRestore}%
}{\endMakeFramed}

% Usage: \begin{takeawaybox} ... \end{takeawaybox}
\newenvironment{takeawaybox}{%
    \def\FrameCommand{\fboxrule=1pt \fboxsep=4pt \fcolorbox{black}{black!5}}%
    \MakeFramed{\advance\hsize-\width \FrameRestore}%
}{\endMakeFramed}

% --- Circled Numbers ---
\newcommand*\circled[1]{\tikz[baseline=(char.base)]{%
    \node[shape=circle,fill,minimum size=0.3cm,inner sep=0.5pt] (char) {\textcolor{white}{#1}};}}

% ============================================================
% AUTHOR COMMENTS
% Colored margin notes for collaborative drafting.
% Hidden when \publicversiontrue is set in config.tex.
% ============================================================
\ifpublicversion
    % --- Production mode: suppress all comments ---
    \newcommand{\grumbler}[3]{}
    \newcommand{\authorA}[1]{}
    \newcommand{\authorB}[1]{}
    \newcommand{\authorC}[1]{}
    \newcommand{\authorD}[1]{}
    \newcommand{\authorE}[1]{}
    \newcommand{\todo}[1]{}
\else
    % --- Drafting mode: show colored inline comments ---
    \newcommand{\grumbler}[3]{\xspace{\textcolor{#3}{\sf\bfseries\small [#1: #2]}}\xspace}
    %
    % Rename these to your co-authors:
    %   \authorA{comment} -> [Alice: comment] in teal
    \newcommand{\authorA}[1]{\grumbler{Alice}{#1}{teal}}
    \newcommand{\authorB}[1]{\grumbler{Bob}{#1}{purple}}
    \newcommand{\authorC}[1]{\grumbler{Carol}{#1}{orange}}
    \newcommand{\authorD}[1]{\grumbler{Dave}{#1}{brown}}
    \newcommand{\authorE}[1]{\grumbler{Eve}{#1}{olive}}
    \newcommand{\todo}[1]{\grumbler{TODO}{#1}{red}}
\fi
